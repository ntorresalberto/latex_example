% !TEX root = ../main.tex

\chapter{State-Space Representation}

One of the main interests of the current work is to find a compact
representation of a robot's dynamics that correctly reflects the kinodynamics
constraints in its actuators:

\begin{align}
  \module{\q} & \leq \qub \\
  \module{\qdot} & \leq \qdotub \\
  \module{\tautorque} & \leq \tauub \\
  \module{\tautorquedot} & \leq \taudotub
                           \shortintertext{where} & \nonumber \\
  \tautorque,\tautorquedot &: \text{represent the joint torque and its derivative} \nonumber
\end{align}

Given ${\poselog=[\pos\ \rotlog]^T \in \sethree}$, where ${\pos
  \incoordspace{R}{3}, \rotlog \in\sothree}$, and joint configuration,
velocity and torques: ${\q,\qdot,\tautorque \incoordspace{R}{n}}$ in a n-dof
serial robot, we can use a system state ${\x=[\poselog\ \q\ \qdot\
  \tautorque]^T}$, along with the
joint torque rate as input ${\uinput=\tautorquedot \incoordspace{R}{n}}$, to arrive at a compact
representation of the system:

\begin{gather}
  \xdot = \A \x + \B \uinput + \V \label{eqn:ss_generic} \\
  \label{eqn:ss_continuous}
  \underbrace{\left[
      \begin{array}{c}
        \twist \\
        \qdot \\
        \qddot \\
        \tautorquedot
      \end{array}\right]}_{\xdot}=
  \underbrace{\left[
      \begin{array}{ccccc}
          & 0 & 0 & \jac & 0 \\
         &&& \eye{n} & 0 \\
          &&&& \imat^{-1} \\
          &&&& 0
      \end{array}\right]}_{\A}
  \underbrace{\left[
      \begin{array}{c}
        \poselog \\
        \q \\
        \qdot \\
        \tautorque
      \end{array}\right]}_{\x}
  + \underbrace{
    \left[
      \begin{array}{ccc}
        0 \\
        0 \\
        0 \\
        \eye[ n ]
      \end{array}\right]}_{\B}
  \underbrace{\tautorquedot_{k}}_{\uinput}
  + \underbrace{
    \left[
      \begin{array}{ccc}
        0 \\
        0 \\
        -\imat^{-1}\biasforces \\
        0
      \end{array}\right]}_{\V}
\end{gather}

\section{Acceleration Observers}

\section{Joint Acceleration Observer}

To measure the joint acceleration, we define a discrete state-space observer can
be designed each time-step as:

\begin{gather}
  \y_k = \C_k\x_k + \D_k \uinput_k + \W_k \label{eqn:ss_discrete_observer}
\end{gather}

Using~\labelcref{eqn:dynmodel}, and treating the term $-\imat^{-1}\biasforces$
as noise, we arrive at the observer expression in the form of~\labelcref{eqn:ss_discrete_observer}:

\begin{gather}
  \underbrace{\qddot_k}_{\y_k} = \underbrace{\left[ \begin{array}{cccc} 0&0&0&\imat^{-1}\end{array}\right]}_{\C_k}
  \x_k +
  \underbrace{\left[ \begin{array}{c} 0\end{array}\right]}_{\D_k}
  \uinput_k +
  \underbrace{-\imat^{-1}\biasforces}_{\W_k}
 \label{eqn:ss_joint_acceleration_observer}
\end{gather}



\section{Operational-space Observer}
In order to track the state of the end-effector, a discrete state-space observer
can be designed with the same form as~\labelcref{eqn:ss_discrete_observer}, this time with $\y = [\pos\ \posdot\ \posddot]^T$ representing the end-effector position
and its derivatives.

Assuming the $\jacdot\qdot$ term in $\posddot=\jac \qddot + \jacdot \qdot$ can
be treated as noise, and using~\labelcref{eqn:dynmodel}, we arrive at a matricial formulation:

\begin{gather}
  \label{eqn:ss_ee_observer}
  \underbrace{\left[
      \begin{array}{c}
        \pos_k \\
        \posdot_k \\
        \posddot_k
      \end{array}\right]}_{\y_k}=
  \underbrace{\left[
      \begin{array}{cccc}
        \eye{d} & 0 & 0 & 0 \\
                && \jac & 0  \\
                &&& \jac \imat^{-1}
      \end{array}\right]}_{\C_k}
  \underbrace{\left[
      \begin{array}{c}
        \pos_k\\
        \q_k\\
        \qdot_k\\
        \tautorque_k
      \end{array}\right]}_{\x_k}
  + \underbrace{\left[
      \begin{array}{c}
        0
      \end{array}\right]}_{\D_k}
  \underbrace{\tautorquedot_{k}}_{\uinput_k}
  + \underbrace{\left[
      \begin{array}{c}
        0 \\
        0 \\
        -\jac\imat^{-1}\biasforces + \jacdot \qdot
      \end{array}\right]}_{\W_k}
\end{gather}

\section{Discretizing the Current System}
\label{sec:discretization_lti_current_system}

Given $\A_c$ in~\labelcref{eqn:ss_continuous} is nillpotent from power 3, we can find
exact expressions for $\A_d,\K, \B_d$ by using their taylor series:

\begin{align}
  \A_d &= \underbrace{\eye[ n ]}_{k=0} + \underbrace{\left[
         \begin{array}{cccc}
           0 & 0 & \jac T_s & 0 \\
           0 & 0 & \eye[ n ] T_s & 0 \\
             &&& \imat^{-1} T_s \\
             &&& 0
         \end{array}\right]}_{k=1} + \underbrace{\left[
                 \begin{array}{cccc}
                   0 & 0 & 0 & \jac \imat^{-1} \frac{T_s^2}{2}\\
                   0 & & 0 & \imat^{-1} \frac{T_s^2}{2} \\
                     &&& 0 \\
                     &&& 0
                 \end{array}\right]}_{k=2} \\
  \Rightarrow \A_d &= \left[
                     \begin{array}{cccc}
                       \eye[ n ] & 0 & \jac T_s & \jac \imat^{-1} \frac{T_s^2}{2} \\
                               & \eye[ n ] & \eye[ n ] T_s & \imat^{-1} \frac{T_s^2}{2} \\
                               &&\eye[ n ]& \imat^{-1} T_s \\
                               &&& \eye[ n ]
                     \end{array}\right]
\end{align}

\begin{align}
  \K &= \underbrace{\eye[ n ] T_s}_{k=1, \text{ }\frac{A_c^0 T_s^1}{1}} + \underbrace{\left[
       \begin{array}{cccc}
         0 & 0 & \jac \frac{T_s^2}{2} & 0 \\
         0 & 0 & \eye[ n ] \frac{T_s^2}{2} & 0 \\
           &&& \imat^{-1} \frac{T_s^2}{2} \\
           &&& 0
       \end{array}\right]}_{k=2, \text{ }\frac{A_c T_s^2}{2}} + \underbrace{\left[
               \begin{array}{cccc}
                 0 & 0 & 0 & \jac \imat^{-1} \frac{T_s^3}{6} \\
                   & & 0 & \imat^{-1} \frac{T_s^3}{6} \\
                   &&& 0 \\
                   &&& 0
               \end{array}\right]}_{k=2,\text{ }\frac{A_c^2 T_s^3}{6}} \\
  \Rightarrow \K &= \left[
                   \begin{array}{cccc}
                     \eye[ n ]T_s & 0 & \jac \frac{T_s^2}{2} & \jac \imat^{-1} \frac{T_s^3}{6} \\
                                & \eye[ n ]T_s & \eye[ n ] \frac{T_s^2}{2} & \imat^{-1} \frac{T_s^3}{6} \\
                                &&\eye[ n ]T_s& \imat^{-1} \frac{T_s^2}{2} \\
                                &&& \eye[ n ]T_s
                   \end{array}\right]
\end{align}

With:
\begin{align}
  \Rightarrow \B_d&= \K \B_c \\
                  &= \left[
                    \begin{array}{c}
                      \jac \imat^{-1} \frac{T_s^3}{6} \\
                      \imat^{-1} \frac{T_s^3}{6} \\
                      \imat^{-1} \frac{T_s^2}{2} \\
                      \eye[ n ]T_s
                    \end{array}\right] \\
  \Rightarrow \V_d&= \K \V_c \\
                  &= \left[
                    \begin{array}{c}
                      -\frac{T_s^2}{2} \jac \imat^{-1}\biasforces \\
                      -\frac{T_s^2}{2} \imat^{-1}\biasforces \\
                      -T_s \imat^{-1}\biasforces \\
                      0
                    \end{array}\right] \\
\end{align}


\section{Generalizing Cartesian-space Tracking with Observer in a receding horizon}

Using the end-effector observer (~\labelcref{eqn:ss_ee_observer}) with an adequately
designed weight matrix $\Q$, we
can formulate the MPC tracking strategy as a quadratic problem of minimizing the
squared distance between $\y$ and the references trajectory points $\y^*$ for
the whole horizon (developped in~\Cref{sec:leastsquare_qp}):
\begin{gather}
  \underset{\z}{\text{min }J} = ||\yoover(\z) - \yoover^*||^2_{\Q} \label{eqn:mpc_cost}
\end{gather}
s.t.
\begin{align}
  \xover &= \Ahat \xhat + \Bhat \uinputhat + \Vhat \label{eqn:ss_z_equality} \\
         &= \left[
           \begin{array}{cc}
             \Ahat & \Bhat
           \end{array}\right]
                     \left[\begin{array}{c}
                             \F_{\xhat} \\
                             \F_{\uinputhat}
                           \end{array}\right] \z + \Vhat \\
  \yoover(\z) &= \Coover \xoover + \Doover \uinputoover + \Woover \label{eqn:ss_z_observer} \\
         &= \left[
           \begin{array}{cc}
             \Coover & \Doover
           \end{array}\right]
                     \left[\begin{array}{c}
                             \F_{\xoover} \\
                             \F_{\uinputoover}
                           \end{array}\right] \z + \Woover \\
  \shortintertext{where:}
  \y_k &= [\pos_k,\posdot_k,\posddot_k]^T \\
  \x_k &= [\pos_k,\qdot_k,\qddot_k,\tautorque_k]^T \\
  \xhat &= \F_{\xhat} \z  = [\x_0, \x_1,...,\x_{H-1}]^T \\
  \uinputhat &= \F_{\uinputhat} \z = [\uinput_0, \uinput_1,...,\uinput_{H-1}]^T \\
  \yhat &= [\y_0, \y_1,...,\y_{H-1}]^T \\
  \yhat^*&= [\y_0^*, \y_1^*,...,\y_{H-1}^*]^T \\
  \xover &= \F_{\xover} \z = [\x_1, \x_2,...,\x_H]^T \\
  \yover &= [\y_1, \y_2,...,\y_H]^T \\
  \xoover &= \F_{\xoover} \z = [\x_1, \x_2,...,\x_{H-1}]^T \\
  \uinputoover &= \F_{\uinputoover} \z = [\uinput_1, \uinput_2,...,\uinput_{H-1}]^T \\
  \yoover &= [\y_1, \y_2,...,\y_{H-1}]^T \\
  \Vhat &= [\underbrace{\V_0, \V_0,...,\V_0}_{H}]^T, \What = [\underbrace{\W_0, \W_0,...,\W_0}_{H}]^T
\end{align}

\textbf{About the Equality Constraint}

To clarify~\labelcref{eqn:ss_z_equality}, it's developed below:

\begin{gather}
  \smaller
  \xover = \Ahat \xhat + \Bhat \uinputhat + \Vhat \tag*{~\labelcref{eqn:ss_z_equality}  revisited} \\
  -\Vhat = \Ahat \xhat + \Bhat \uinputhat -\xover \\
  -\Vhat = \left[
    \begin{array}{ccc}
      \Ahat & \Bhat & -\eye[ n_{\xover} ]
    \end{array}\right]
  \left[\begin{array}{c}
          \F_{\xhat} \\
          \F_{\uinputhat} \\
          \F_{\xover}
        \end{array}\right] \z
\end{gather}

\begin{gather}
        -\underbrace{\left[
          \begin{array}{c}
            \V \\
            \V \\
            \vdots \\
            \V
          \end{array}\right]}_{\Vhat} =
      \underbrace{\left[
          \begin{array}{c|c|c}
            \begin{array}{ccccc}
              \A\\
              &\A\\
              &&\ddots\\
              &&&\A
            \end{array}
              & \begin{array}{ccccc}
                  \B\\
                  &\B\\
                  &&\ddots\\
                  &&&\B
                \end{array}
              & \begin{array}{c}
                  -\eye[ n_{\xover} ]
                \end{array}
          \end{array}\right]}_{\left[ \begin{array}{c|c|c} \Ahat & \Bhat & -\eye[ n_{\xover} ] \end{array}\right]}
      \underbrace{\left[
          \begin{array}{ccc}
            \eye[ n_{\xhat}\times n_{\xhat} ]
            & \zeros[ n_{\xhat}\times n_x ]
            & \zeros[ n_{\xhat}\times n_{\uinputhat} ] \\
            \hline
            \zeros[ n_{\uinputhat}\times n_{\xhat} ]
            & \zeros[ n_{\uinputhat}\times n_x ]
            & \eye[ n_{\uinputhat}\times n_{\uinputhat} ] \\
            \hline
            \zeros[ n_{\xover}\times n_x ]
            & \eye[ n_{\xover}\times n_{\xover} ]
            & \zeros[ n_{\xover}\times n_{\uinputhat} ]
          \end{array}\right]}_{\left[\begin{array}{c}
                                       \F_{\xhat} \\
                                       \hline
                                       \F_{\uinputhat}\\
                                       \hline
                                       \F_{\xover}
                                     \end{array}\right]}
      \underbrace{\left[
          \begin{array}{c}
            \x_0 \\
            \x_1 \\
            \x_2 \\
            \vdots \\
            \x_{H-1} \\
            \x_H \\
            \uinputhat
          \end{array}\right]}_{\z} \\
      \shortintertext{with:} \nonumber \\
      n_z = n_x+n_{\xhat}+n_{\uinputhat} \\
      n_{\xhat} = Hn_x \\
      n_{\uinputhat} = Hn_{\uinput}
\end{gather}

\textbf{About the observer Cost Function}

It's important to note in~\labelcref{eqn:ss_ee_observer}, $\D=0$ (and thus in $\Doover$ in~\labelcref{eqn:ss_z_observer}),  to make it more clear:

\begin{gather}
  \yoover(\z) = \Coover \xoover + \Doover \uinputoover + \Woover \tag*{~\labelcref{eqn:ss_z_observer}  revisited} \\
  \yoover(\z) = \Coover \underbrace{\F_{\xoover} \z}_{\xoover} + \Woover \\
  \smaller
  \underbrace{\left[
      \begin{array}{c}
        \y_1 \\
        \y_2 \\
        \y_3 \\
        \vdots \\
        \x_{H-1}
      \end{array}\right]}_{\yoover}=
  \underbrace{\left[
      \begin{array}{ccccc}
        \C \\
        & \C \\
        & & \C \\
        &&&\ddots \\
        & & & & \C \\
      \end{array}\right]}_{\Coover}
  \underbrace{\left[
      \begin{array}{ccccc}
        \zeros[ n_{\xoover}\times n_x ]
        & \eye[ n_{\xoover}\times n_{\xoover} ]
        & \eye[ n_{\xoover}\times n_x ]
        & \zeros[ n_{\xoover}\times n_{\uinput} ]
        & \zeros[ n_{\xoover}\times n_{\uinputoover} ]
      \end{array}\right]}_{\F_{\xoover}}
  \underbrace{\left[
      \begin{array}{c}
        \x_0 \\
        \xoover \\
        \x_H \\
        \uinput_0 \\
        \uinputoover
      \end{array}\right]}_{\z}
  +
  \underbrace{\left[
      \begin{array}{c}
        \W \\
        \W \\
        \vdots \\
        \W
      \end{array}\right]}_{\Woover} \\
  \shortintertext{with:} \nonumber \\
  n_z = n_x+n_{\xoover}+n_x+n_{\uinputhat} \\
  n_{\uinputhat} = n_{\uinputoover} + n_{\uinput} \\
  n_{\uinputoover} = (H-1)n_{\uinput} \\
  n_{\xoover} = (H-1)n_x
\end{gather}


\section{Using the generalized observer for position tracking}
For an adequately chosen $\Q$, we can extract the end-effector pose and
define que the cost function~\labelcref{eqn:mpc_cost} in such a way that:

\begin{gather}
  \underset{\z}{\text{min }J} = ||\posoover(\z) - \posoover^*||^2_{\Q} \label{eqn:mpc_pos_cost}
\end{gather}

where $\posoover = [\pos_1,\pos_2, ..., \pos_{H-1}]^T$, and likewise for the
reference trajectory in the horizon~$\posoover^*$.

As for $\Q$, it's defined in such a way that:
\begin{gather}
  (\yoover - \yoover^*)^T\Q(\yoover - \yoover^*) =
  (\posoover - \posoover^*)^T\Q(\posoover - \posoover^*) \\
  \Q = \left[
    \begin{array}{ccccc}
      \F_{\pos} \\
      & \F_{\pos} \\
      & & \F_{\pos} \\
      &&&\ddots \\
      & & & & \F_{\pos} \\
    \end{array}\right] \\
\shortintertext{with:} \nonumber\\
\pos_k = \F_{\pos} \y_k \\
\F_{\pos}=
\left[
  \begin{array}{ccc}
    \eye[ n_d\times n_d ] \\
    & \zeros[ n_d\times n_d ] \\
    & & \zeros[ n_d\times n_d ]
  \end{array}\right]
\end{gather}

Where $n_d$ is the dimension of the workspace.
