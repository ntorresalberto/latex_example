% \pdfcompresslevel=0
% \pdfobjcompresslevel=0

\newcommand{\withbigtikz}[0]{} % uncomment to include c5 graphs
% \newcommand{\csvfilesuffix}[0]{fast}
% \newcommand{\csvfilesuffix}[0]{spline}
\newcommand{\csvfilesuffix}[0]{normal}

\usepackage{amsmath,bm} % bm redefines \boldsymbol{Symbol}!
% \usepackage{setspace} % allows spacing in arrays?


%%%%%%%%%%%%%%%%%%%%%%%%%%%%%%%%%%%%%%%%%%%%%%%%%%%%%
% shrink font size locally
% (fixes problem with caption font size in tikz sub-figures)
% https://tex.stackexchange.com/questions/250457/font-scaling-shrink-all-fontsizes-locally
\usepackage{geometry}
\makeatletter
\let\zzfontsize\fontsize
\def\zz#1#2{{%
    \def\fontsize##1##2{%
      \@defaultunits\@tempdima##1pt\relax\@nnil
      \@defaultunits\@tempdimb##1pt\relax\@nnil
      \zzfontsize{#1\@tempdima}{#1\@tempdimb}}#2}}
\makeatother
% NOTE: 0.64 is the magic number for font size footnotesize
\newcommand{\captiontikz}[2]{\zz{0.64}{\caption{#1}\label{#2}}}
% \newcommand{\captiontikz}[1]{\caption{#1}}
%%%%%%%%%%%%%%%%%%%%%%%%%%%%%%%%%%%%%%%%%%%%%%%%%%%%%
% image page with different margins
% https://tex.stackexchange.com/a/78285
\usepackage{afterpage}
%%%%%%%%%%%%%%%%%%%%%%%%%%%%%%%%%%%%%%%%%%%%%%%%%%%%%

\usepackage{tabularx}


%%%%%%%%%%%%%%%%
% table configs: https://tex.stackexchange.com/a/176780
\usepackage{makecell}
\renewcommand\theadalign{bc}
\renewcommand\theadfont{\bfseries}
\renewcommand\theadgape{\Gape[4pt]}
\renewcommand\cellgape{\Gape[4pt]}
%%%%%%%%%%%%%%%%


\usepackage{dashrule}

\usepackage{marginnote}
\usepackage{qrcode}
% \usepackage{pst-barcode}
% \usepackage{auto-pst-pdf}

\usepackage{amssymb}




%%%%%%%%%%%%%%%%%%%%%%%%%%%%%%%%%%%%%%%%%%%%%%%%%%%%%
% force right margin notes
% https://tex.stackexchange.com/a/599049
\usepackage{ifthen,changepage}
\usepackage{xargs}
\newcommandx{\leftmarginnote}[2][2=0pt]
{\checkoddpage
  \ifoddpage
    {\reversemarginpar\marginnote{#1}[#2]}
  \else
    {\marginnote{#1}[#2]}
  \fi}
\newcommandx{\rightmarginnote}[2][2=0pt]
{\checkoddpage
  \ifoddpage
    {\marginnote{#1}[#2]}
  \else
    {\reversemarginpar\marginnote{#1}[#2]}
  \fi}
%%%%%%%%%%%%%%%%%%%%%%%%%%%%%%%%%%%%%%%%%%%%%%%%%%%%%



%%%%%%%%%%%%%%%%%%%%%%%%%%%%%%%%%%%%%%%%%%%%%%%%%%%%%
% revisited equations in marginnote
% https://tex.stackexchange.com/questions/25491/manual-customizable-reference-text
% \newcommand{\futurev}[1]{\marginnote{\footnotesize (revisited~\hyperref[#1]{later})}[0cm]}
\newcommand{\revhspace}[0]{\hspace{.3cm}}
\newcommand{\futurev}[1]{{
    \scriptsize
    \rightmarginnote{\revhspace
      \hyperref[#1]{{\footnotesize revisited}~$\downarrow$}
    }[0cm]}}
\newcommand{\prevrev}[1]{{
    \scriptsize
    \rightmarginnote{\revhspace
      \hyperref[#1]{{\footnotesize revisited}~$\uparrow$}
    }[0cm]}}
\newcommand{\prevfuturev}[2]{{
    \scriptsize
    \rightmarginnote{\revhspace
      \hyperref[#1]{{\footnotesize
          revisited}~$\uparrow$}~~\hyperref[#2]{$\downarrow$}
    }[0cm]}}
%%% for debugging:
% \newcommand{\futurev}[1]    {\rightmarginnote{\scriptsize revisited \labelcref{#1}}[0cm]}
% \newcommand{\prevrev}[1]    {\rightmarginnote{\scriptsize revisited \labelcref{#1}}[0cm]}
% \newcommand{\prevfuturev}[2]{\rightmarginnote{\scriptsize revisited \labelcref{#1},\labelcref{#2}}[0cm]}
%%%%%%%%%%%%%%%%%%%%%%%%%%%%%%%%%%%%%%%%%%%%%%%%%%%%%


%%%%%%%%%%%%%%%%%%%%%%%%%%%%%%%%%%%%%%%%%%%%%%%%%%%%%
% chapter appendices
% https://tex.stackexchange.com/questions/120716/appendix-after-each-chapter
% \usepackage{appendix}
\usepackage[page,toc,titletoc,title]{appendix}
\AtBeginEnvironment{subappendices}{%
  \chapter*{Appendix}
  \addcontentsline{toc}{chapter}{Appendices}
  \counterwithin{figure}{section}
  \counterwithin{table}{section}
}

%%%%%%%%%%%%%%%%%%%%%%%%%%%%%%%%%%%%%%%%%%%%%%%%%%%%%
% used for theorems and definitions
% Conflict with ICRA FORMAT
% amsthm.sty:441: Command \proof already defined.
\usepackage[thmmarks, thref, amsmath]{ntheorem}%
% \usepackage{amsthm}
% \theoremstyle{definition}
% \newtheorem{definition}{Definition}[section]

% https://tex.stackexchange.com/questions/396485/is-there-a-way-to-modify-the-theorem-environment-in-order-to-have-an-horizontal
% \theoremstyle{plain}
% \theoremprework{\bigskip\hrule\vspace{-1.5ex}\leavevmode\nobreak
% \leavevmode
% }
%   \theorempostwork{
%   \vspace*{2mm}
%   %   \vspace{-1ex}
%   \hrule\bigskip\leavevmode
% }
\theoremheaderfont{\bfseries \scshape}
\theorembodyfont{}%{\itshape}
\theoremseparator{. }

\theoremprework{
    \vspace*{-2mm}
    \begin{minipage}{\textwidth}
    \bigskip\rule{\textwidth}{1pt}
    % \bigskip\hrule
    % \vspace*{2mm}
  }
  \theorempostwork{
    % \vspace*{2mm}
    % \hrule\bigskip
    \rule{\textwidth}{1pt}\bigskip
  \end{minipage}
  \vspace*{-2mm}
}
\newtheorem*{remark}{Remark}


%%%%%%%%%%%%%%%%%%%%%%%%%%%%%%%%%%%%%%%%%%%%%%%%%%%%%
% make toc back and forth when clicking sections
% https://tex.stackexchange.com/a/460820
\usepackage{etoolbox}
\usepackage{hyperref}
\makeatletter
\patchcmd{\addcontentsline}{{#2}{#3}{\thepage }}%
{{#2}{#3}{\protect\Hy@raisedlink{\protect\hypertarget{back\@currentHref}{}}{\thepage}} }%
{\typeout{** patch \string\addcontentsline\space success}}%
{\typeout{** patch \string\addcontentsline\space failure}}%
\AtBeginDocument{\renewcommand{\Sectionformat}[2]%
  {\raggedright % avoids hyphenation (word breaks/splitting) in section/subsections
    \ifnum #2>\c@secnumdepth {#1}\else\hyperlink{back\@currentHref}{#1~{\scriptsize$\uparrow$}}\fi}}
\patchcmd{\@makechapterhead}{\bfseries #1\par}{\bfseries\hyperlink{back\@currentHref}{#1}\par}%
{\typeout{** patch \string\@makechapterhead\space success}}%
{\typeout{** patch \string\@makechapterhead\space failure}}
\makeatother

% https://tex.stackexchange.com/a/498546
\AtEndPreamble{\RequirePackage{hyperref}
  \hypersetup{
    hypertexnames=true,
    colorlinks=true,
    % linkcolor=blue
    citecolor=blue
  }
}

%%%%%%%%%%%%%%%%%%%%%%%%%%%%%%%%%%%%%%%%%%%%%%%%%%%%%

% inspirational quote
% https://tex.stackexchange.com/a/53378
\usepackage{epigraph}
\setlength{\epigraphwidth}{0.63\textwidth}



\usepackage{soul} % for striked out text (\st)
\usepackage{subcaption}
% \captionsetup[subfigure]%
% {labelformat=empty,justification=RaggedRight}
\captionsetup[subfigure]%
{
  subrefformat=simple,labelformat=simple,
  justification=justified, labelfont={rm, footnotesize}, font=footnotesize, width=0.5\textwidth}
\captionsetup[figure]%
{justification=justified, labelfont={rm, footnotesize}, font=footnotesize,
  width=1\textwidth, margin=1cm}
\usepackage{adjustbox}
\captionsetup[table]%
{justification=justified, labelfont={rm, footnotesize}, font=footnotesize,
  width=1\textwidth, margin=1cm}



% https://tex.stackexchange.com/a/180382
\renewcommand\thesubfigure{(\alph{subfigure})}


% \captionsetup{
%   justification=justified,
%   % font={scriptsize,sf},
%   % font={small},
%   font={footnotesize},
%   width=1.1\textwidth} % Adjust the width as desired


%%%%%%%%%%%%%%%%%%%%%%%%%%%%%%%%%%%%%%%%%%%%%%

% required for tikzplotlib
% \usepackage[utf8]{inputenc}
\usepackage{pgfplots}
% More defined colors

\usepackage{multicol}
\usepackage{tikz}
\usepackage{tikz-3dplot}
% \usepackage[dvipsnames]{xcolor}
\usetikzlibrary{positioning}

\usepackage{robotarm}
% \DeclareUnicodeCharacter{2212}{−}
\usepgfplotslibrary{groupplots,dateplot}
\usepgfplotslibrary{fillbetween}
\usetikzlibrary{patterns,shapes,arrows}

\pgfplotsset{compat=newest,width=0.90\textwidth,height=130pt}
% \pgfplotstableread[col sep = comma]{result.csv}\resultscsv
% \pgfplotstableread[col sep = comma]{result_fast.csv}\resultscsv
\definecolor{brown}{RGB}{165,42,42}
\definecolor{dimgray85}{RGB}{229,229,229}
\definecolor{firebrick}{RGB}{178,34,34}
\definecolor{forestgreen}{RGB}{34,139,34}
\definecolor{gainsboro229}{RGB}{255,255,255}
\definecolor{steelblue}{RGB}{70,130,180}
\definecolor{limits_color2}{RGB}{211,211,211}


\definecolor{midnightblue}{rgb}{0.1, 0.1, 0.44}
\definecolor{forestgreen}{rgb}{0.13, 0.55, 0.13}
\definecolor{maroon}{rgb}{0.5, 0.0, 0.0}

\colorlet{mydarkblue}{blue!50!black}
\colorlet{xcol}{blue!70!black}
\colorlet{xcol'}{xcol!50!red!80!black}
\colorlet{veccol}{green!45!black}

\usetikzlibrary{intersections}
\usetikzlibrary{decorations.markings}
\usetikzlibrary{decorations.pathmorphing}
\usetikzlibrary{angles,quotes} % for pic
\usetikzlibrary{math}


% \tikzset{>=latex} % for LaTeX arrow head
% \tikzset{font={\fontsize{10pt}{12}\selectfont}}
% \pgfplotsset{major grid style={dotted,white!50!black}}

\tikzstyle{vector}=[->,thick,veccol,line cap=round]
\tikzstyle{rvec}=[->,thick,xcol,line cap=round]

\tikzstyle{xvec}=[->,very thick,firebrick]
\tikzstyle{yvec}=[->,very thick,forestgreen]
\tikzstyle{zvec}=[->,very thick,steelblue!90!black]
\tikzstyle{xvecu}=[->,very thick,firebrick!80!black, cap=round]
\tikzstyle{yvecu}=[->,very thick,forestgreen!80!black, cap=round]
\tikzstyle{zvecu}=[->,very thick,steelblue!70!black, cap=round]

% \usetikzlibrary{external}


\usepgfplotslibrary{external}
\tikzexternalize[%
mode=list and make, % https://tex.stackexchange.com/questions/40516/externalization-to-other-format-makefile-add-new-rules-to-the-makefile
prefix=pgfplotsfigures/]% Folder needs to be created before compiling

% \tikzexternalize[%
% up to date check={simple},
% prefix=pgfplotsfigures/]% Folder needs to be created before compiling

\tikzset{external/system call={%
    pdflatex -halt-on-error
    % --save-size=80000
    % --pool-size=10000000
    % --extra-mem-top=50000000
    % --extra-mem-bot=50000000
    % --main-memory=90000000
    -file-line-error
    % --interaction=batchmode
    --interaction=nonstopmode
    --jobname "\image" "\texsource"}}

% \tikzset{external/system call={%
%     pdflatex \tikzexternalcheckshellescape
%     -enable-write18 -halt-on-error -shell-escape -interaction=batchmode -output-directory=./pgfplotsfigures
%     -jobname "\image" "\texsource"}}

% \usepackage{pdfpages}
% \tikzexternalize[shell escape=-enable-write18,optimize command away=\includepdf]



%%%%%%%%%%%%%%%%%%%%%%%%%%%%%%%%%%%%%%%%%%%%%%

        \tikzstyle{block} = [draw, align=center, fill=steelblue!20, rectangle, minimum height=1.3cm, minimum width=2cm]
\tikzstyle{block_control} = [draw, align=center, fill=steelblue!30, rectangle, minimum height=1.3cm, minimum width=2cm]
  \tikzstyle{block_model} = [draw, align=center, fill=firebrick!30, rectangle, minimum height=1.3cm, minimum width=2cm]
  \tikzstyle{block_env} = [draw, align=center, fill=forestgreen!30, rectangle, minimum height=1.3cm, minimum width=2cm]
\tikzstyle{sum} = [draw, align=center, fill=steelblue!20, circle, node distance=1cm]
\tikzstyle{input} = [coordinate]
\tikzstyle{output} = [coordinate]
\tikzstyle{pinstyle} = [pin edge={to-,thin,black}]
\tikzstyle{snakearrow} = [->,decorate,decoration={snake, amplitude=.7mm, segment
  length=2mm, pre length=.5mm, post length=1mm}]
\tikzstyle{doublesnakearrow} = [decorate, <->,
decoration={snake, amplitude=1.5mm, segment length=4mm, pre length=2mm, post
  length=2mm}]

% \usepackage{cleveref} % for range of equations
% \newcommand{\crefrangeconjunction}{--}% for range of equations



% https://tex.stackexchange.com/a/121055
\usepackage[nameinlink]{cleveref}
\AtBeginEnvironment{appendices}{\crefalias{chapter}{appendix}}
% see for equations: https://tex.stackexchange.com/a/647018


% see for equations: https://tex.stackexchange.com/a/647018
% \setlength{\marginparwidth}{2cm}
% \usepackage{todonotes}
\usepackage{mathtools}
% \newtagform{show_eq}{(Eq.\ }{)}  % how the equation numbers are displayed
% \newtagform{show_eq}{(}{)}  % how the equation numbers are displayed
% \usetagform{show_eq}

% \renewcommand{\theequation}{\textit{Eq. \thesection.\arabic{equation}}}
% \renewcommand{\theequation}{\arabic{section}.\arabic{equation}}
% \renewcommand{\theequation}{\arabic{equation}} % uses numbering in arabic
                                % instead of roman

% % removes section/subsection from equation numbering
% \usepackage{chngcntr}
% \numberwithin{equation}{section} % how the equation numbers are formed
% \counterwithout{equation}{chapter}
% \counterwithout{equation}{section} % don't reset eq counter each Section
% \counterwithout{equation}{subsection}

\setcounter{secnumdepth}{5}

% https://texblog.org/2012/12/21/multi-column-and-multi-row-cells-in-latex-tables/
\usepackage{multirow}

\usepackage[ruled,vlined,linesnumbered]{algorithm2e} % for formatted algorithms
\SetKwFor{For}{for (}{) $\lbrace$}{$\rbrace$}
\SetNlSty{textbf}{}{:}% Add colon after line number
\IncMargin{.2em}% Push algorithm to the right (allowing for larger line numbering)

\usepackage[makeroom]{cancel} % example: cross out tends to infinity


\usepackage{xparse} % needed for \IfNoValueTF (logic in function widhout args)

\newcommand{\circled}[1]{\raisebox{.5pt}{\textcircled{\raisebox{-.9pt} {#1}}}}



% see for equations: https://tex.stackexchange.com/a/647018
% \renewcommand{\eqref}[1]{eq. \textup{\ref{eqn:#1}}}
% \renewcommand{\eqref}[1]{~(\textup{\ref{eqn:#1}})}
%\newcommand{\figref}[1]{fig. \textup{\ref{fig:#1}}}
%\newcommand{\algref}[1]{alg. \textup{\ref{alg:#1}}}

% How to disable hyphenation in all section and subsection titles
% https://tex.stackexchange.com/a/24783
% \usepackage[raggedright]{titlesec} % breaks back ref

% \usepackage{titlesec}
% \titleformat{\paragraph}
% {\normalfont\normalsize\bfseries}{\theparagraph}{1em}{}
% \titlespacing*{\paragraph}
% {0pt}{3.25ex plus 1ex minus .2ex}{1.5ex plus .2ex}
% https://tex.stackexchange.com/questions/8361/latex-ieeetran-cls-use-titlesec-package
%\newcommand{\subparagraph}{}

\usepackage[normalem]{ulem} % strikethrough text with \sout

\newcommand{\eqarray}[1]{\begin{cases}
                           #1
                         \end{cases}}

% interval/vectors
\newcommand\X[0]{\boldsymbol{X}}
\newcommand\Y[0]{\boldsymbol{Y}}
\newcommand\y[0]{\boldsymbol{y}}
\newcommand\yhat[0]{\boldsymbol{\hat{y}}}
\newcommand\yover[0]{\boldsymbol{\overline{y}}}
\newcommand\yoover[0]{\boldsymbol{\overline{\overline{y}}}}
\newcommand\Yhat[0]{\boldsymbol{\hat{Y}}}
\newcommand\B[0]{\boldsymbol{B}} % bias forces (coriolis + gravity)
\newcommand\Bunder[0]{\boldsymbol{\underline{B}}}
\newcommand\Bover[0]{\boldsymbol{\overline{B}}}
\newcommand\Bhat[0]{\boldsymbol{\hat{B}}}
\newcommand\C[0]{\boldsymbol{C}}
\newcommand\Chat[0]{\boldsymbol{\hat{C}}}
\newcommand\Cover[0]{\boldsymbol{\overline{C}}}
\newcommand\Cunder[0]{\boldsymbol{\underline{C}}}
\newcommand\Coover[0]{\boldsymbol{\overline{\overline{C}}}}
\newcommand\ubold[0]{\boldsymbol{u}}
\newcommand\U[0]{\boldsymbol{U}}
\newcommand\Uhat[0]{\boldsymbol{\hat{U}}}
\newcommand\Uover[0]{\boldsymbol{\overline{U}}}
\newcommand\Uunder[0]{\boldsymbol{\underline{U}}}
\newcommand\E[0]{\boldsymbol{E}}
\newcommand\A[0]{\boldsymbol{A}}
\newcommand\Ahat[0]{\boldsymbol{\hat{A}}}
\newcommand\Aover[0]{\boldsymbol{\overline{A}}}
\newcommand\Aunder[0]{\boldsymbol{\underline{A}}}
\newcommand\Aoover[0]{\boldsymbol{\overline{\overline{A}}}}
\newcommand\D[0]{\boldsymbol{D}} % incompatible with ARK22 template
\newcommand\Dhat[0]{\boldsymbol{\hat{D}}}
\newcommand\Dover[0]{\boldsymbol{\overline{D}}}
\newcommand\Doover[0]{\boldsymbol{\overline{\overline{D}}}}
\newcommand\Q[0]{\boldsymbol{Q}}
\newcommand\Qdot[0]{\boldsymbol{\dot{Q}}}
\newcommand\Qddot[0]{\boldsymbol{\ddot{Q}}}
\newcommand\Qdddot[0]{\boldsymbol{\dddot{Q}}}
\newcommand\Wcal[0]{\mathcal{W}}
\newcommand\Lcal[0]{\mathcal{L}}
\newcommand\Lcaldot[0]{\mathcal{\dot{L}}}
\newcommand\Lcalddot[0]{\mathcal{\ddot{L}}}
\newcommand\Lcaldddot[0]{\mathcal{\dddot{L}}}
\newcommand\Qcal[0]{\mathcal{Q}}
\newcommand\Qcaldot[0]{\mathcal{\dot{Q}}}
\newcommand\Qcalddot[0]{\mathcal{\ddot{Q}}}
\newcommand\Qcaldddot[0]{\mathcal{\dddot{Q}}}
\newcommand\Qcalkyn[0]{\Qcal_{\kyn}}

\newcommand\kk[0]{\boldsymbol{k}}
\newcommand\K[0]{\boldsymbol{K}}
\newcommand\R[0]{\boldsymbol{R}}
\newcommand\Rdot[0]{\boldsymbol{\dot{R}}}
\newcommand\Pbold[0]{\boldsymbol{P}}
\newcommand\boldgamma[0]{\boldsymbol{\gamma}}
\newcommand\boldgammaunder[0]{\underline{\boldsymbol{\gamma}}}
\newcommand\V[0]{\boldsymbol{V}}
\newcommand\Vcal[0]{\mathcal{\boldsymbol{V}}}
\newcommand\Ucal[0]{\mathcal{\boldsymbol{U}}}
\newcommand\Hcal[0]{\mathcal{\boldsymbol{H}}}
\newcommand\Ccal[0]{\mathcal{\boldsymbol{C}}}
\newcommand\Acal[0]{\mathcal{\boldsymbol{A}}}
\newcommand\Scal[0]{\mathcal{\boldsymbol{S}}}
\newcommand\Mcal[0]{\mathcal{\boldsymbol{M}}}
\newcommand\Ycal[0]{\mathcal{\boldsymbol{Y}}}
\newcommand\Tcal[0]{\mathcal{\boldsymbol{T}}}
\newcommand\Tcaldot[0]{\mathcal{\boldsymbol{\dot{T}}}}
\newcommand\Vhat[0]{\boldsymbol{\hat{V}}}
\newcommand\Vunder[0]{\boldsymbol{\underline{V}}}
\newcommand\W[0]{\boldsymbol{W}}
% \newcommand\E[0]{\boldsymbol{E}} % incompatible with ARK22 template
\newcommand\What[0]{\boldsymbol{\hat{W}}}
\newcommand\Wover[0]{\boldsymbol{\overline{W}}}
\newcommand\Woover[0]{\boldsymbol{\overline{\overline{W}}}}
\newcommand\F[0]{\boldsymbol{F}}
\newcommand\boldxiunder[0]{\boldsymbol{\underline{\xi}}}
\newcommand\boldxi[0]{\boldsymbol{\xi}}
\newcommand\boldXi[0]{\boldsymbol{\Xi}}
\newcommand{\irchi}[2]{\raisebox{\depth}{$#1\chi$}} % inner command, used by \rchi
% \DeclareRobustCommand{\rchi}{{\mathpalette\irchi\relax}}
% \newcommand\boldchi[0]{\boldsymbol{\rchi}}
% \newcommand\boldchihat[0]{\boldsymbol{\hat{\rchi}}}
% \newcommand\boldChi[0]{\scalebox{1.2}{$\boldsymbol{\rchi}$}}
\newcommand\boldChi[0]{\boldsymbol{\mathcal{X}}}
\newcommand\lift[0]{\boldsymbol{\psi}}



% \newcommand\taskconstraintssym[0]{\mathrm{\Gamma}}
% \newcommand\taskconstraintssym[0]{\mathcal{X}}
\newcommand\taskconstraintssym[0]{\mathcal{P}}
\newcommand\taskconstraintsgeom[0]   {\taskconstraintssym}
\newcommand\taskconstraintsgeomd[0]  {\mathcal{V}}
\newcommand\taskconstraintsgeomdd[0] {\mathcal{A}}
\newcommand\taskconstraintsgeomddd[0]{\mathcal{J}}
\newcommand\taskconstraintskyn[0]{\taskconstraintssym_{\kyn}}
\newcommand\Bcal[0]{\mathcal{B}}
\newcommand\pol[0]{\mathcal{P}}
\newcommand\polhat[0]{\mathcal{\hat{P}}}
\newcommand\pold[0]{\dot{\mathcal{P}}}
\newcommand\poldd[0]{\ddot{\mathcal{P}}}
\newcommand\polddd[0]{\dddot{\mathcal{P}}}

\newcommand\kyn[0]{\mathcal{K}}

\newcommand\Nt[0]{\text{N}}
\newcommand\kt[0]{\text{k}}
\newcommand\horizonsteps[0]{\text{h}}
\newcommand\initialpose[0]{\pose_i}
\newcommand\initialposelog[0]{\poselog_i}
\newcommand\targetstate[0]{\x_t}
\newcommand\targetx[0]{\x_t}
\newcommand\targetXover[0]{\Xover_t}
\newcommand\targetpose[0]{\pose_t}
\newcommand\targetposelog[0]{\poselog_t}
\newcommand\targetpos[0]{\pos_t}
\newcommand\targetR[0]{\R_t}
\newcommand{\ad}[1][]{%
  \def\FirstArg{#1}
  \ifx\FirstArg\empty
  % if #1 is empty
  \ensuremath{\text{\textbf{ad}}}
  \else
  % if #1 is not empty
  \ensuremath{\text{\textbf{ad}}_{#1}}
  \fi
  \:
}
\newcommand{\Ad}[1][]{%
  \def\FirstArg{#1}
  \ifx\FirstArg\empty
  % if #1 is empty
  \ensuremath{\text{\textbf{Ad}}}
  \else
  % if #1 is not empty
  \ensuremath{\text{\textbf{Ad}}_{#1}}
  \fi
  \:
}


\newcommand\jointint[1]{\boldsymbol{q_{#1}}}
\newcommand\lbold[0]{\boldsymbol{l}}
\newcommand\f[0]{\boldsymbol{f}}
\newcommand\g[0]{\boldsymbol{g}}
\newcommand\q[0]{\boldsymbol{q}}
\newcommand\qhat[0]{\hat{\q}}
\newcommand\deltatvec[0]{\boldsymbol{\Delta t}}
\newcommand\dt[0]{\delta t}
\newcommand\deltalog[0]{\boldsymbol{\delta}}


\newcommand\lb[0]{\bbold_{\lbold}}
\newcommand\ub[0]{\bbold_{\ubold}}
\newcommand\Alb[0]{\A_{\lbold}}
\newcommand\Aub[0]{\A_{\ubold}}
\newcommand\Clb[0]{\C_{\lbold}}
\newcommand\Cub[0]{\C_{\ubold}}
\newcommand\qlb[0]{\q_{m}}
\newcommand\qub[0]{\q_{M}}
\newcommand\qdotlb[0]{\qdot_{m}}
\newcommand\qdotub[0]{\qdot_{M}}
\newcommand\qddotlb[0]{{\qddot}_{m}}
\newcommand\qddotub[0]{{\qddot}_{M}}
\newcommand\qdddotlb[0]{{\qdddot}_{m}}
\newcommand\qdddotub[0]{{\qdddot}_{M}}
\newcommand\taulb[0]{{\tautorque}_{m}}
\newcommand\tauub[0]{{\tautorque}_{M}}
\newcommand\taudotlb[0]{{\tautorquedot}_{m}}
\newcommand\taudotub[0]{{\tautorquedot}_{M}}

\newcommand\p[0]{\boldsymbol{p}}
\newcommand\abold[0]{\boldsymbol{a}}
\newcommand\bbold[0]{\boldsymbol{b}} % bias forces (coriolis + gravity)
\newcommand\cbold[0]{\boldsymbol{c}}
\newcommand\dbold[0]{\boldsymbol{d}}
\newcommand\ebold[0]{\boldsymbol{e}}
\newcommand\gbold[0]{\boldsymbol{g}}
\newcommand\hbold[0]{\boldsymbol{h}}
\newcommand\rbold[0]{\boldsymbol{r}}

\newcommand\vel[0]{\boldsymbol{v}}
\newcommand\acc[0]{\boldsymbol{a}}
\newcommand\jer[0]{\boldsymbol{j}}

% \newcommand\vel[0]{\posdot}
% \newcommand\acc[0]{\posddot}
% \newcommand\jer[0]{\posdddot}


\newcommand\w[0]{\boldsymbol{w}}
\newcommand\x[0]{\boldsymbol{x}}
\newcommand\xhat[0]{\boldsymbol{\hat{x}}}
\newcommand\xhor[1]{\boldsymbol{x}_{#1}}
\newcommand\xtilde[0]{\boldsymbol{\tilde{x}}}
\newcommand\xdothat[0]{\boldsymbol{\dot{\hat{x}}}}
\newcommand\xddothat[0]{\boldsymbol{\ddot{\hat{x}}}}
\newcommand\xover[0]{\boldsymbol{\overline{x}}}
\newcommand\xoover[0]{\boldsymbol{\overline{\overline{x}}}}
\newcommand\Xhat[0]{\boldsymbol{\hat{X}}}
\newcommand\Xover[0]{\boldsymbol{\overline{X}}}
\newcommand\Xunder[0]{\boldsymbol{\underline{X}}}
\newcommand\Xunderover[0]{\boldsymbol{\underline{\overline{X}}}}
\newcommand\xbar[0]{\boldsymbol{\bar{x}}}
\newcommand\z[0]{\boldsymbol{z}}
\newcommand\zhat[0]{\boldsymbol{\hat{z}}}
\newcommand\xdot[0]{\boldsymbol{\dot{x}}}
\newcommand\xddot[0]{\boldsymbol{\ddot{x}}}
\newcommand\xdddot[0]{\boldsymbol{\dddot{x}}}
\newcommand\Xdot[0]{\boldsymbol{\dot{X}}}
\newcommand\Xddot[0]{\boldsymbol{\ddot{X}}}
\newcommand\ydot[0]{\boldsymbol{\dot{y}}}
\newcommand\zdot[0]{\boldsymbol{\dot{z}}}
\newcommand\e[0]{\boldsymbol{e}}
\newcommand\edot[0]{\boldsymbol{\dot{e}}}
\newcommand\eddot[0]{\boldsymbol{\ddot{e}}}

\definecolor{ForestGreen}{RGB}{34,139,34}
\definecolor{Brown}{RGB}{121,37,0}
% \definecolor{Sepia}{RGB}{103,24,0}
\definecolor{BitterSweet}{RGB}{192,79,23}

% \newcommand\wip[1]{\textcolor{BitterSweet}{\textbf{#1}}}
\newcommand{\wip}[1]{{\leavevmode\color{BitterSweet}#1}}

\newcommand\colorize[1]{\textcolor{red}{#1}}
\newcommand\colorizee[1]{\textcolor{orange}{#1}}
\newcommand\colorizeee[1]{\textcolor{blue}{#1}}
\newcommand\colorizeeee[1]{\textcolor{ForestGreen}{#1}}
% https://tex.stackexchange.com/questions/65058/how-to-change-the-fboxsep-and-fboxrule-of-ltxexamples-rframe
\fboxrule=0.2ex % sets thickness of boxes
\newcommand\mynote[1]{\noindent\fbox{\begin{minipage}{0.47\textwidth}#1\end{minipage}}}
% \newcommand\mycolornote[1]{\noindent\fcolorbox{orange}{white}{\begin{minipage}{0.47\textwidth}#1\end{minipage}}}
\newcommand\mycolornote[1]{\noindent\fcolorbox{BitterSweet}{white}{
% \minipage[t]{0.47\textwidth}#1
\minipage[t]{\textwidth}#1
\endminipage}}
\newcommand\boxeditems[1]{\mycolornote{\items{#1}}}


\newcommand\items[1]{{
    % \emergencystretch 3em %https://tex.stackexchange.com/a/9110
    \begin{itemize} #1
    \end{itemize}}
}
\newcommand\enumerated[1]{{
    % \emergencystretch 3em %https://tex.stackexchange.com/a/9110
    \begin{enumerate} #1 \end{enumerate}}
}
\newcommand\error[0]{\boldsymbol{e}}
\newcommand\poss[0]{p}
\newcommand\pos[0]{\boldsymbol{p}}
\newcommand\posdot[0]{\boldsymbol{\dot{p}}}
\newcommand\posddot[0]{\boldsymbol{\ddot{p}}}
\newcommand\posdddot[0]{\boldsymbol{\dddot{p}}}
\newcommand\poshat[0]{\boldsymbol{\hat{p}}}
\newcommand\phat[0]{\hat{\p}}
\newcommand\velhat[0]{\hat{\vel}}
\newcommand\acchat[0]{\hat{\acc}}
\newcommand\posover[0]{\boldsymbol{\overline{p}}}
\newcommand\posoover[0]{\boldsymbol{\overline{\overline{p}}}}
\newcommand\posdothat[0]{\boldsymbol{\hat{\dot{p}}}}
\newcommand\posdotover[0]{\boldsymbol{\overline{\dot{p}}}}
\newcommand\posdotoover[0]{\boldsymbol{\overline{\overline{\dot{p}}}}}
\newcommand\posddothat[0]{\boldsymbol{\hat{\ddot{p}}}}
\newcommand\posdddothat[0]{\boldsymbol{\hat{\dddot{p}}}}
\newcommand\posddotover[0]{\boldsymbol{\overline{\ddot{p}}}}
\newcommand\posddotoover[0]{\boldsymbol{\overline{\overline{\ddot{p}}}}}
\newcommand\uinput[0]{\boldsymbol{u}}
\newcommand\uinputdot[0]{\boldsymbol{\dot{u}}}
\newcommand\uinputddot[0]{\boldsymbol{\ddot{u}}}
\newcommand\uinputhat[0]{\boldsymbol{\hat{u}}}
\newcommand\uinputtilde[0]{\boldsymbol{\tilde{u}}}
\newcommand\uinputunder[0]{\boldsymbol{\underline{u}}}
\newcommand\uinputover[0]{\boldsymbol{\overline{u}}}
\newcommand\uinputoover[0]{\boldsymbol{\overline{\overline{u}}}}
\newcommand\uinputhor[1]{\boldsymbol{u}_{#1}}
\newcommand\T[0]{\boldsymbol{T}}
\newcommand\Tdot[0]{\boldsymbol{\dot{T}}}
\newcommand\qdot[0]{\boldsymbol{\dot{q}}}
\newcommand\qddot[0]{\boldsymbol{\ddot{q}}}
\newcommand\qdddot[0]{\boldsymbol{\dddot{q}}}

% quaternion
\newcommand\quat[0]{r}
\newcommand\quati[1]{r_{#1}}
\newcommand\quatreal[0]{\quat_w}
\newcommand\quatrealdot[0]{\dot{\quat_w}}
\newcommand\quatim[0]{\boldsymbol{\quat_v}}
\newcommand\quatimdot[0]{\boldsymbol{\dot{\quat_v}}}
\newcommand\quatvec[0]{\boldsymbol{\quat}}
\newcommand\quatvecdot[0]{\boldsymbol{\dot{\quat}}}
\newcommand\qi[0]{\boldsymbol{i}}
\newcommand\qj[0]{\boldsymbol{j}}
\newcommand\qk[0]{\boldsymbol{k}}

\newcommand\tautorque[0]{\boldsymbol{\tau}}
\newcommand\tautorquehat[0]{\boldsymbol{\hat{\tau}}}
\newcommand\tautorquep[0]{\tautorque'}
\newcommand\tautorquedot[0]{\boldsymbol{\dot{\tau}}}
\newcommand\tautorquedothat[0]{\boldsymbol{\dot{\hat{\tau}}}}
\newcommand\tautorquekp[0]{\tautorque'_k}
\newcommand\tautorquekpt[0]{{\tautorquekp}^T}

%\newcommand\skewmat[1]{[#1]_{\scalebox{0.7}{$\times$}}}
\newcommand\skewmat[1]{{#1}^{^{\wedge}}\,}
\newcommand\unskewmat[1]{{#1}^{^{\vee}}\,}

\newcommand\maxx[0]{{\text{max}}}
\newcommand\minn[0]{{\text{min}}}
\newcommand\m[0]{{\text{m}}}
\newcommand\ms[0]{{\text{ms}}}
\newcommand\rad[0]{{\text{rad}}}
\newcommand\s[0]{{\text{s}}}
\newcommand\dexp[0]{{\text{dexp}}}
\newcommand\dlog[0]{{\text{dlog}}}
\newcommand\Jlog[0]{{\text{J}^{\log}_\text{R}}}
\newcommand\angular[0]{{\text{ang}}}
\newcommand\linear[0]{{\text{linear}}}
\newcommand\lin[0]{{\text{lin}}}
\newcommand\mpc[0]{{\text{mpc}}}
\newcommand\target[0]{{\text{t}}}
\newcommand\optt[0]{{\text{opt}}}



\newcommand\bch[0]{\boldsymbol{\Omega}}
\newcommand\bchf[1]{\bch(#1)} %linear velocity
\newcommand\linvel[0]{\boldsymbol{v}} %linear velocity
\newcommand\linveldot[0]{\boldsymbol{\dot{v}}} %linear velocity
\newcommand\linvelddot[0]{\boldsymbol{\ddot{v}}} %linear velocity

\newcommand\boldeps[0]{\boldsymbol{\epsilon}}
\newcommand\boldepsdot[0]{\boldsymbol{\dot{\epsilon}}}
\newcommand\rotlognorm[0]{\phi} %rotational log
\newcommand\rotlog[0]{\boldsymbol{\rotlognorm}} %rotational log
\newcommand\rotloghat[0]{\boldsymbol{\hat{\rotlognorm}}} %rotational log
\newcommand\rotvel[0]{\boldsymbol{\omega}} %rotational velocity
\newcommand\rotveldot[0]{\boldsymbol{\dot{\omega}}} %rotational velocity
\newcommand\rotvelddot[0]{\boldsymbol{\ddot{\omega}}} %rotational velocity
\newcommand\rotvelhat[0]{\boldsymbol{\hat{\omega}}} %rotational velocity
\newcommand\rotvell[0]{\omega} %rotational velocity
\newcommand\rotacc[0]{\boldsymbol{\alpha}} %rotational acceleration
\newcommand\motiond[0]{\hat{\linvel}}
\newcommand\linacc[0]{\boldsymbol{a}}
\newcommand\jac[0]{\boldsymbol{J}} %jacobian
\newcommand\jachat[0]{\boldsymbol{\hat{J}}} %jacobian
\newcommand\jacdot[0]{\boldsymbol{\dot{J}}}
\newcommand\jacddot[0]{\boldsymbol{\ddot{J}}}
\newcommand\jacquat[0]{\jac_\quat}
% \newcommand\twist[0]{\boldsymbol{T}}
\newcommand\twist[0]{\boldsymbol{\nu}}
% \newcommand\twist[0]{\boldsymbol{\mathcal{V}}}
\newcommand\twistd[0]{\boldsymbol{\dot{\nu}}}
\newcommand\twistacc[0]{\boldsymbol{a}}

\newcommand\twistdd[0]{\boldsymbol{\ddot{\nu}}}
\newcommand\initialtwist[0]{\twist_i}
% \newcommand\twist[0]{\boldsymbol{\upsilon}}

\newcommand\rhobold[0]{\boldsymbol{\rho}}
\newcommand\rhobolddot[0]{\boldsymbol{\dot{\rho}}}
\newcommand\rhoboldddot[0]{\boldsymbol{\ddot{\rho}}}
\newcommand\pose[0]{\mathcal{X}}
\newcommand\posedot[0]{\mathcal{\dot{X}}}
\newcommand\poselog[0]{\boldxi}
\newcommand\poselogdot[0]{\rhobolddot}
\newcommand\poselogddot[0]{\rhoboldddot}
\newcommand\oq[0]{\quatvec} % orientation quaternion
\newcommand\oqdot[0]{\dot{\quatvec}} % orientation quaternion


\newcommand\iden[1]{\boldsymbol{\mathrm{I}}_{#1}}
\newcommand\eye[1][]{\iden{#1}}
\newcommand\zeros[1][]{\boldsymbol{0_{#1}}}

% robot dynamics
\newcommand\wrench[0]{\boldsymbol{W}}
\newcommand\imat[0]{\boldsymbol{M}} %inertia matrix
% \usepackage{physics}
% \renewcommand{\imat}[0]{\boldsymbol{M}} % package physics defines imat for the identity
\newcommand\imatcart[0]{\boldsymbol{\Lambda}} %inertia matrix cartesian space
\newcommand\imatcartv[0]{\imatcart_{\linvel}} %linear inertia matrix cartesian space
\newcommand\imatcartw[0]{\imatcart_{\rotvel}} %rotational inertia matrix cartesian space
\newcommand\biasforces[0]{\boldsymbol{b}} % bias forces (coriolis + gravity)
\newcommand\biasforcesdot[0]{\boldsymbol{\dot{B}}} % bias forces (coriolis + gravity)
\newcommand\cor[0]{\boldsymbol{C}} % coriolis
\newcommand\grav[0]{\boldsymbol{G}} % gravity


\newcommand{\lieG}[0]{\mathcal{G}}
\newcommand{\lieg}[0]{\mathfrak{g}}
\newcommand{\liespace}[1]{\mathfrak{#1}^{3}}
\newcommand{\coordspace}[2]{\mathbb{#1}^{#2}}
\newcommand{\incoordspace}[2]{\in \coordspace{#1}{#2}}
\newcommand{\module}[1]{\left\vert#1\right\vert}
% \newcommand{\norm}[1]{\left\lVert#1\right\rVert}
\newcommand{\SOn}[1]{\mathbb{SO}(#1)}
\newcommand{\sothree}[0]{\mathfrak{so}(3)}
\newcommand{\SOthree}[0]{\mathbb{SO}(3)}
\newcommand{\sotwo}[0]{\mathfrak{so}(2)}
\newcommand{\SOtwo}[0]{\mathbb{SO}(2)}
\newcommand{\setwo}[0]{\mathfrak{se}(2)}
\newcommand{\SEtwo}[0]{\mathbb{SE}(2)}
\newcommand{\sethree}[0]{\mathfrak{se}(3)}
\newcommand{\SEthree}[0]{\mathbb{SE}(3)}
% https://www.overleaf.com/learn/latex/Spacing_in_math_mode
% \newcommand{\suchthat}[0]{\;\:|\;\:}
% \newcommand{\suchthat}[0]{\,|\,}
\newcommand{\suchthat}[0]{~|~}
\newcommand{\argminn}[1]{\underset{#1}{\arg\min}}
\DeclareMathOperator*{\argmin}{\arg\min}
%\NewDocumentCommand{\norm}{ O{2} O{2} m  }{\left\|#3\right\|^{{}^{#2}}_{{}_{#1}}}
\newcommand{\reg}[0]{\text{reg}}
\newcommand{\reft}[0]{\text{ref}}
\newcommand{\des}[0]{\text{des}}
\newcommand{\opt}[0]{\text{opt}}
\newcommand{\manif}[0]{\mathcal{M}}
\newcommand{\leftbr}[0]{\left[\!\!}
\newcommand{\rightbr}[0]{\!\!\right]}
\newcommand{\Ttext}[0]{\text{T}}




\newcommand{\rvect}[1]{\begin{bmatrix} #1 \end{bmatrix}}
\newcommand{\vvect}[1]{\begin{bmatrix} #1 \end{bmatrix}}
\newcommand{\interval}[1]{[\underline{#1}, \overline{#1}]}

\maxdeadcycles=500
\usepackage{graphicx}
% \usepackage{cleveref}
% \crefname{plot}{plot}{plots}
% \Crefname{plot}{Plot}{Plots}
% \usepackage{subcaption}
% \usepackage[style=base]{caption}
% \newcommand*\SubplotLabel[1]{
%         \node [
%             anchor=north west,
%             text width=2em,
%             text height=2ex,
%             text depth=1ex,
%             align=left,
%         ] at (axis description cs:0.02,0.98)
%             {\phantomsubcaption\label{#1}\subref{#1})};
%     }

% \newcommand{\SubplotLabel}[1]{%
% \node [below right] at (rel axis cs:0,1) {\refstepcounter{plot}\label{#1}\ref{#1})};
% }



\newcommand{\insertfigure}[1]{\noindent\makebox[\textwidth]{\includegraphics[width=0.8\paperwidth]{figures/#1}}}


\newcommand{\figurecaptionlabel}[3]{
  \begin{figure}
    \centerline{\includegraphics[width=0.8\paperwidth]{figures/#1}}
    \IfNoValueTF{#2}{
      % no label
    } {\caption{#2}}

    % labeling
    \IfNoValueTF{#3}{
      % no label
    } {
      % label provided
      \label{fig:#3}
    }
  \end{figure}
}

\newcommand{\myequation}[1]{
  \begin{equation}
    \begin{aligned}
      #1
    \end{aligned}
  \end{equation}
}
\newcommand{\myrefequation}[2]{\myequation{\label{eqn:#1}#2}}

\newcommand{\myarray}[2]{\begin{array}{#1} #2 \end{array}}
\newcommand{\mymat}[2]{\left[\begin{array}{#1} #2 \end{array}\right]}


% \setlength{\parindent}{4em}
% \setlength{\parskip}{1em}
\usepackage{standalone}
\usepackage{listings} % for code highlight
% \lstset{
%   escapeinside={\%*}{*)}          % if you want to add LaTeX within your code
% }

\newcommand{\diag}[1]{\mathrm{diag}\left(#1\right)}

% \usepackage[
% sort,
% section,
% numberedsection=autolabel,
% nogroupskip,
% abbreviations,
% automake]{glossaries-extra}
% \usepackage[abbreviations, automake]{glossaries-extra} % breaks tikzexternalize?
\usepackage[abbreviations]{glossaries-extra}
% https://tex.stackexchange.com/a/435680
\preto\chapter{\glsresetall}
\preto\section{\glsresetall}
\makeglossaries
\newabbreviation{ts}{TS}{Technical Specification}
\newabbreviation{hri}{HRI}{Human-Robot Interaction}
\newabbreviation{mpc}{MPC}{Model Predictive Control}
\newabbreviation{ss}{SS}{State Space}
\newabbreviation{fk}{FK}{Forward Kinematics}
\newabbreviation{ik}{IK}{Inverse Kinematics}
\newabbreviation{pid}{PID}{Proportional Integral Derivative}
\newabbreviation{me}{ME}{Magnus Expansion}
\newabbreviation{qp}{QP}{Quadratic Program}
\newabbreviation{pd}{PD}{Proportional Derivative}
\newabbreviation{csc}{CSC}{Cartesian-space Constraints}
\newabbreviation{jsc}{JSC}{Joint-space Constraints}
\newabbreviation{lti}{LTI}{Linear Time-Invariant}
\newabbreviation{dof}{DoF}{Degree of Freedom}


\newcommand{\figwidth}{0.8\linewidth}
\newcommand{\smallfigwidth}{0.5\linewidth}

\addtolength{\jot}{5pt} % math interline spacing
\renewcommand{\arraystretch}{1.2} % array spacing



% https://tex.stackexchange.com/questions/4637/correct-use-of-paragraph-titles
\let\originalparagraph\paragraph
\renewcommand{\paragraph}[2][.]{\originalparagraph{#2#1}}
% If you don't want a period, you can use something like \paragraph[?]{Title}, then a '?' will be placed at the end instead of a period.


%%%%%%%%%%%%%%%%%%%%%%%%%%%%%%%%%%%%%%%%%%%%%%%%%
% https://tex.stackexchange.com/a/59139
\makeatletter
\newenvironment{summary}
{ % \addcontentsline{toc}{section}{Summary}{}{}
  % \begin{center}\textbf{Abstract}\end{center}
  \list{}{\listparindent 1em%
    % \setlength{\leftmargin}{<value>} adjust if you need
    \itemindent\listparindent
    \rightmargin\leftmargin
    \parsep\z@ \@plus\p@}%
\item\relax}
{\noindent\rule{0.618\textwidth}{0.7pt}
  \endlist}
%%%%%%%%%%%%%%%%%%%%%%%%%%%%%%%%%%%%%%%%%%%%%%%%%

%%%%%%%%%%%%%%%%%%%%%%%%%%%%%%%%%%%%%%%%%%%%%%%%%%%%%
% https://tex.stackexchange.com/a/2437
% workaround to avoid loading this in test file
\usepackage{ifthen}
\ifthenelse{\isundefined{\testfile}}{

%%%%%%%%%%%%%%%%%%%%%%%%%%%%%%%%%%%%%%%%%%%%%%%%%%%%%
\usepackage{etoc}
\usepackage{tocloft}
\tocloftpagestyle{empty}

% \renewcommand{\cftchapfont}{\bfseries\LARGE}
% \renewcommand{\cftsecfont}{\bfseries\Large}
% \renewcommand{\cftsubsecfont}{\large}
\renewcommand{\cftsubsubsecfont}{\small}

\renewcommand{\cftchapfont}{\bfseries\large}
\renewcommand{\cftsecfont}{\bfseries\normalsize}
% \renewcommand{\cftsubsecfont}{\small}
% \renewcommand{\cftsubsubsecfont}{\small}


% \renewcommand\cftchapafterpnum{\vskip 3pt}
\setlength{\cftbeforechapskip}{4pt}

% \renewcommand\cftsecafterpnum{\vskip 0pt}
\setlength{\cftbeforesecskip}{1pt}

\renewcommand\cftsubsecafterpnum{\vskip 1pt}
% \setlength{\cftbeforesubsecskip}{1pt}


% \setlength{\cftsecindent}{0em}
% \setlength{\cftsubsecindent}{0em}
% \setlength{\cftsubsubsecindent}{0em}

% \setlength{\cftsecnumwidth}{3.5em}
% \setlength{\cftsubsecnumwidth}{3.5em}
% \setlength{\cftsubsubsecnumwidth}{3.5em}

% https://tex.stackexchange.com/a/297657
% \usepackage[figure,table]{totalcount}
% \iftotalfigures
%   \listoffigures
% \fi
% \iftotaltables
%   \listoftables
% \fi
%%%%%%%%%%%%%%%%%%%%%%%%%%%%%%%%%%%%%%%%%%%%%%%%%%%%%

}{} % ifthen workaround to avoid loading this in test file

% horizontal / landscape pages
% https://tex.stackexchange.com/a/354
% \usepackage{pdflscape}
\usepackage{lscape}
% \begin{landscape}
%   ...
% \end{landscape}

\usepackage{afterpage}
% \afterpage{
% }

\usepackage{lipsum}
